% Document preamble
\documentclass[11pt]{article}
\usepackage[margin=1in]{geometry}
\usepackage[pdftex]{graphicx}
\usepackage{multirow}
\usepackage{setspace}
\usepackage{hyperref}
\usepackage{float}
\restylefloat{table}
\pagestyle{plain}
\setlength{\parindent}{0cm}
%%%%%%%%%%%%%%%%%%%%%%%%%%%%%%%%%%%%%%%%%%%%%%%%%%%%%%%%%%%%%%%%%%%%%%%%%%%%%%%%

\begin{document}

%%%%%%%%%%%%%%%%%%%%%%%%%%%%%%%%%%%%%%%%%%%%%%%%%%%%%%%%%%%%%%%%%%%%%%%%%%%%%%%%
% Course information
%%%%%%%%%%%%%%%%%%%%%%%%%%%%%%%%%%%%%%%%%%%%%%%%%%%%%%%%%%%%%%%%%%%%%%%%%%%%%%%%
\begin{tabular}{ l l }
  \multirow{3}{*}{\includegraphics[height=1.25in,
    width=1.25in]{figs/ucberkeleyseal_874_540.eps}}
  & \LARGE Public Health 290 --- Spring 2018 Syllabus\\
  & \LARGE Targeted Learning with Biomedical Big Data \\\\
  & \Large Class meets TuTh TBA in TBA \\
  & \Large Lab meets W 2:00--3:00P in 230 Mulford \\\\
  & \Large Course Control Number: TBA \\
\end{tabular}
\vspace{10mm}

%%%%%%%%%%%%%%%%%%%%%%%%%%%%%%%%%%%%%%%%%%%%%%%%%%%%%%%%%%%%%%%%%%%%%%%%%%%%%%%%
% Instructor information
%%%%%%%%%%%%%%%%%%%%%%%%%%%%%%%%%%%%%%%%%%%%%%%%%%%%%%%%%%%%%%%%%%%%%%%%%%%%%%%%
\begin{tabular}{ l l }
  \multirow{6}{*} & \large Instructor: Mark van der Laan \\
  & \large E-mail: \href{mailto:laan@berkeley.edu}{laan@berkeley.edu} \\
  & \large Web: \url{https://vanderlaan-group.github.io} \\
  & \large Office Location: 108 Haviland Hall \\
  & \large Office Hours: Tu 4:00--5:00P \\
\end{tabular}
\quad
\begin{tabular}{ l l }
  \multirow{6}{*} & \large GSI: Nima Hejazi \\
  & \large E-mail: \href{mailto:nhejazi@berkeley.edu}{nhejazi@berkeley.edu} \\
  & \large Web: \url{https://nimahejazi.org} \\
  & \large Office Location: 111 Haviland Hall \\
  & \large Office Hours: M 11:00A--12:00P \\
\end{tabular}
\vspace{5mm}
\begin{center} The instructional staff reserve the right to make changes to the
  syllabus at any time.\\
\end{center}

%%%%%%%%%%%%%%%%%%%%%%%%%%%%%%%%%%%%%%%%%%%%%%%%%%%%%%%%%%%%%%%%%%%%%%%%%%%%%%%%
% Course details
%%%%%%%%%%%%%%%%%%%%%%%%%%%%%%%%%%%%%%%%%%%%%%%%%%%%%%%%%%%%%%%%%%%%%%%%%%%%%%%%
\textbf {\large \\ Course Description:}
This course teaches students how to construct efficient estimators and obtain
robust inference for parameters that utilize data-adaptive estimation
strategies (i.e., machine learning). Students will perform hands-on
implementation of novel estimators using high-dimensional biomedical data sets,
providing students with a toolbox for analyzing complex longitudinal,
observational, and randomized control trial data. Students will actively learn
and apply the core principles of the Targeted Learning methodology, which (1)
generalizes machine learning to any estimand of interest; (2) obtains an optimal
estimator of the given estimand, grounded in theory; (3) integrates the
state-of-the-art ensemble machine learning techniques; and (4) provides formal
statistical inference in terms of confidence intervals and testing of specified
null hypotheses of interest. It also integrates causal inference thereby
allowing one to define estimands that represent the answer to causal questions
of interest. \\

%%%%%%%%%%%%%%%%%%%%%%%%%%%%%%%%%%%%%%%%%%%%%%%%%%%%%%%%%%%%%%%%%%%%%%%%%%%%%%%%
% Instructional strategy
%%%%%%%%%%%%%%%%%%%%%%%%%%%%%%%%%%%%%%%%%%%%%%%%%%%%%%%%%%%%%%%%%%%%%%%%%%%%%%%%
\textbf {\large \\ Instructional Strategy:}
Most pedagogical studies (i.e., those concerned with the methods and
effectiveness of teaching) indicate that lectures by themselves are a poor way
of engaging students and promoting learning. To address this problem, this
course will use a Blended Learning/Hybrid Classroom Format. This involves
shifting the majority of the material presented in class and out of class.
Instructional core content is delivered online, outside of the classroom. Class
time is spent exploring topics in greater depth and creates meaningful learning
opportunities. This rearrangement allows for more interactive, active learning
opportunities during class time like group discussion, Q\&A, problem solving
activities, and R labs where students will apply the methods presented to real
data. It also allows for self-paced comprehension of highly complex core
concepts. Video lectures give students the ability to pause, rewind, and even
re-watch content delivery opposed to traditional lectures that require content
delivery to occur in a fixed time and place. The video lectures are intended to
serve as jumping off points to drive discussion, activities, and clarification
during class time. Thus, it is essential that students watch the video lecture
and take notes before class. Under this instructional model, coming to class
confused is welcome, but coming to class empty-headed is ineffective. \\

\newpage

%%%%%%%%%%%%%%%%%%%%%%%%%%%%%%%%%%%%%%%%%%%%%%%%%%%%%%%%%%%%%%%%%%%%%%%%%%%%%%%%
% Prerequisites
%%%%%%%%%%%%%%%%%%%%%%%%%%%%%%%%%%%%%%%%%%%%%%%%%%%%%%%%%%%%%%%%%%%%%%%%%%%%%%%%
\textbf {Prerequisites:}
\begin{itemize} \itemsep-0.5em
  \item Public Health C240A / Statistics C245A.
  \item Statistics 201A-B (or older version Statistics 200A-B).
  \item \textit{recommended:} Statistics C239A / Political Science C236A or
    Public Health 252D or equivalent introduction to statistical causal
    inference.
\end{itemize}

\textbf {Credit Hours:} 4 \\

\textbf {\large Text(s):}
There will be assigned and optional readings from the following textbooks that
serve as excellent references, encompassing all of the topics covered in this
course and more:
\begin{itemize}
  \item \textit{Targeted Learning: Causal Inference for Observational and
    Experimental Data} by Mark J. van der Laan and Sherri Rose (2011)
    \{abbreviated to vdL\&R (2011) in the sequel\}
  \item \textit{Targeted Learning in Data Science: Causal Inference for Complex
    Longitudinal Studies} by Mark J. van der Laan and Sherri Rose (2017)
    \{abbreviated to vdL\&R (2017) in the sequel\}
\end{itemize}

These textbooks are freely available in PDF for students to download through
SpringerLink. \\

We encourage the use of Mark's blog for web-based discussion. Students are
invited to submit a question to the blog by sending an email to
\href{mailto:vanderlaan.blog@berkeley.edu}{vanderlaan.blog@berkeley.edu} at any
time. Course assignments will require submission of questions to the blog. \\

Lecture videos, lab activity R scripts, homework assignments, upcoming due
dates, the syllabus, and student grades will be made available via the bCourses
intranet. There will be an anonymous mid-semester feedback survey; the link will
be sent to students' berkeley.edu email addresses. This survey will be provided
through SurveyMonkey. \\

\textbf {\large Course Objectives:} \\
At the completion of this course, students should be able to
\begin{enumerate} \itemsep-0.4em
  \item utilize the R statistical software to do the following to either
    longitudinal, observational, or randomized trial biomedical data:
    \begin{itemize}
      \item apply the Super Learner algorithm to estimate various functional
        target parameters,
      \item implement TML Estimation to construct an efficient estimator of a
        target parameter of interest and obtain statistical inference via
        confidence intervals,
      \item design data simulations to practically evaluate these estimators and
        their inference;
    \end{itemize}
  \item formulate the statistical estimation problem using formal notation by
    defining scientific question in terms of a structural causal model, the type
    of intervention, and the causal parameter of interest;
  \item establish identifiability of the causal parameter from the observed data
    distribution;
  \item understand the framework of the Super Learner approach for the
    estimation of parameters and oracle inequalities for the general
    cross-validation selector;
  \item evaluate the conditions guaranteeing asymptotic linearity and efficiency
    of the TMLE for a diverse set of commonly encountered, real-world estimators
    of interest. \\
\end{enumerate}

\newpage

\textbf {\large Grading:} \\
\begin{itemize}
  \item \textit{Assignments (\textbf{40\%}):} Concern the application of the
    above learning outcomes. Assignments incorporate written problems and
    programming in R. Assignments will be distributed via GitHub Classroom, and
    no late assignments will be accepted. There will be between three and five
    homework assignments.
  \item \textit{Final Project (\textbf{30\%}):} Consists of the presentation of
    a topic that involves the application of statistical methods and software to
    address a particular question of interest.
  \item \textit{Participation (\textbf{20\%}):} Includes engagement during
    class, coming prepared to class, team-based learning exercises, a blog
    question assignment, and course evaluations. Note that preparedness for
    class will be assessed in the form of short, random concept checks that are
    open note and easy for students who listened to the lecture video before
    class.
  \item \textit{Attendance (\textbf{10\%}):} Recorded at all lab and lecture
    meetings. You must email Prof.~van der Laan, not the GSI, if you cannot
    attend class. \\
\end{itemize}

\textbf{\large Course Policies:}

\textbf{Accommodations}:
Please see the instructor as soon as possible if you need particular
accommodations, and we will work out the necessary arrangements. \\

\textbf{Scheduling Conflicts}:
Notify the instructional staff by the second week of the term about any known or
potential conflicts (e.g., religious observances, interviews, team activities,
conferences). \\

\textbf{Collaboration and Independence}:

All homework assignments should clearly list collaborators and references.
Homework assignments will not be considered for credit if they are a replicate
of another classmate's assignment. With that in mind, you may work together but
you should complete answers independently. \\

\textbf{Honor Code}:

``\textit{As a member of the UC Berkeley community, I act with honesty,
integrity, and respect for others}.'' \\

The purpose of the Honor Code is to enhance awareness of the need for the
highest possible levels of integrity and respect on campus, both within and
outside the academic context. We hope and believe that the code will catalyze a
series of ongoing conversations about our principles and practices. Together,
through engagement, we can create a consistent message and ethos in our
classrooms, labs, departments, and throughout the academic enterprise, to ensure
that the core values of academic integrity and honesty are being embraced by
both students and faculty. Please read the Honor
Code (\url{http://asuc.org/honorcode/index.php}) carefully.\\

\textbf{Academic Integrity}:

One of the most important values of an academic community is the balance between
the free flow of ideas and the respect for the intellectual property of others.
Researchers don't use one another's research without permission; scholars and
students always use proper citations in papers; professors may not circulate or
publish student papers without the writer's permission; and students may not
circulate or post materials (handouts, exams, syllabi --- any class materials)
from their classes without the written permission of the instructor.

Any test, paper or report submitted by you and that bears your name is presumed
to be your own original work that has not previously been submitted for credit
in another course unless you obtain prior written approval to do so from
your instructor. In all of your assignments, you may use words or ideas written
by other individuals in publications, web sites, or other sources, but only with
proper attribution. If you are not clear about the expectations for completing
an assignment or taking a test or examination, be sure to seek clarification
from your instructor or GSI beforehand. Finally, you should keep in mind that as
a member of the campus community, you are expected to demonstrate integrity in
all of your academic endeavors and will be evaluated on your own merits. The
consequences of cheating and academic dishonesty—including a formal discipline
file, possible loss of future internship, scholarship, or employment
opportunities, and denial of admission to graduate school—are simply not worth
it. \\

\textbf{Students with disabilities}: If you require accommodations, please make
arrangements in a timely manner through the DSP office.\\

\noindent\textbf{Important Dates}:
\begin{center} \begin{minipage}{5in}
\begin{flushleft}
Homework 1 \dotfill Thursday, Feb.~15\\
Homework 2 \dotfill Thursday, Mar.~01\\
Homework 3 \dotfill Thursday, Mar.~15\\
Final Project Proposal \dotfill Thursday, Mar.~22\\
Midterm Feedback Survey \dotfill Thursday, Mar.~22\\
Homework 4 \dotfill Thursday, Apr.~05\\
Homework 5 \dotfill Thursday, Apr.~19\\
Blog Question \dotfill Friday, Apr.~20\\
Final Project Presentations \dotfill Thursday, May 03\\
Final Project Report \dotfill Friday, May 04\\
\end{flushleft}
\end{minipage}
\end{center}

\newpage

% Course Outline
\textbf {\large Tentative Course Outline}:

The weekly coverage is subject to change, as it depends on the progress of the
class.

\begin{table}[H]
\normalsize % The size of the table text can be changed depending on content.
\begin{tabular}{ | c | c | }
\hline
\textbf{Week} & \textbf{Content} \\
\hline

1: 16--18 Jan. & \begin{minipage}{.85\textwidth}
\begin{itemize} \itemsep-0.4em
  \vspace{1mm}
  \item \textit{Topics:} The roadmap of statistical learning
  \item \textit{Before:} Read course syllabus; watch Week 1 video lectures
  \item \textit{Lab:} Data Science with R: Introduction to the Tidyverse
  \vspace{1mm}
\end{itemize}
\end{minipage} \\
\hline

2: 23--25 Jan. & \begin{minipage}{.85\textwidth}
\begin{itemize} \itemsep-0.4em
  \vspace{1mm}
  \item \textit{Topics:} Examples of data-generating experiments, traditional
    data analysis
  \item \textit{Before:} Read Ch.~1 of vdL\&R (2011); watch Week 2 video
    lectures
  \item \textit{Lab:} Reproucible Research with R, git, and GitHub
  \vspace{1mm}
\end{itemize}
\end{minipage} \\
\hline

3: 30 Jan.--01 Feb. & \begin{minipage}{.85\textwidth}
\begin{itemize} \itemsep-0.4em
  \vspace{1mm}
  \item \textit{Topics:} Structural causal models, causal quantities,
    identification
  \item \textit{Before:} Read Ch.~2 of vdL\&R (2011); watch Week 3 video
    lectures
  \item \textit{Lab:} Structural causal models
  \vspace{1mm}
\end{itemize}
\end{minipage} \\
\hline

4: 06--08 Feb. & \begin{minipage}{.85\textwidth}
\begin{itemize} \itemsep-0.4em
  \vspace{1mm}
  \item \textit{Topics:} Interventions, optimal interventions, and
    identifiability results
  \item \textit{Before:} Watch Week 4 video lectures
  \item \textit{Lab:} Interventions and identifiability
  \vspace{1mm}
\end{itemize}
\end{minipage} \\
\hline

5: 13--15 Feb. & \begin{minipage}{.85\textwidth}
\begin{itemize} \itemsep-0.4em
  \vspace{1mm}
  \item \textit{Topics:} Understanding the challenges of nonparametric density
    estimation: Super Learning of a density
  \item \textit{Before:} Read Ch.~3 of vdL\&R (2011); watch Week 5 video
    lectures
  \item \textit{Lab:} Introduction to Super Learner and the {\rm sl3} R package
  \item \textit{Deliverables:} Homework assignment 1 due 15 Feb.~by 11:59pm
  \vspace{1mm}
\end{itemize}
\end{minipage} \\
\hline

6: 20--22 Feb. & \begin{minipage}{.85\textwidth}
\begin{itemize} \itemsep-0.4em
  \vspace{1mm}
  \item \textit{Topics:} Super Learner and oracle inequality for the general
    cross-validation selector, Super Learning in prediction, Super Learning of
    optimal individualized treatment rules
  \item \textit{Before:} Watch Week 6 video lectures
  \item \textit{Lab:} Super Learner libraries and screening algorithms
  \vspace{1mm}
\end{itemize}
\end{minipage} \\
\hline

7: 27 Feb.--01 Mar. & \begin{minipage}{.85\textwidth}
\begin{itemize} \itemsep-0.4em
  \vspace{1mm}
  \item \textit{Topics:} Super Learning of conditional multinomial distributions
    or densities
  \item \textit{Before:} Read Ch.~15 of vdL\&R (2017); watch Week 7 video
    lectures
  \item \textit{Lab:} Super Learning of densities with the {\rm sl3} and
    {\rm condensier} R packages
  \item \textit{Deliverables:} Homework assignment 2 due 01 Mar.~by 11:59pm
  \vspace{1mm}
\end{itemize}
\end{minipage} \\
\hline

8: 06--08 Mar. & \begin{minipage}{.85\textwidth}
\begin{itemize} \itemsep-0.4em
  \vspace{1mm}
  \item \textit{Topics:} Measure-theoretic Integrtion , the Highly Adaptive
    Lasso (HAL)
  \item \textit{Before:} Read Ch.~6 of vdL\&R (2017); watch Week 8 video
    lectures
  \item \textit{Lab:} The Highly Adaptive Lasso and the {\rm hal9001} R package
  \vspace{1mm}
\end{itemize}
\end{minipage} \\
\hline

9: 13 Mar.--15 Mar. & \begin{minipage}{.85\textwidth}
\begin{itemize} \itemsep-0.4em
  \vspace{1mm}
  \item \textit{Topics:} Asymptotic lineary, influence curves, and inference
    based on influence curves
  \item \textit{Before:} Read A.~1--A.2 of vdL\&R (2011); watch Week 9 video
    lectures
  \item \textit{Lab:} Estimation and inference with (efficient) influence
    functions, part 1
  \item \textit{Deliverables:} Homework assignment 3 due 15 Mar.~by 11:59pm
  \vspace{1mm}
\end{itemize}
\end{minipage} \\
\hline

\end{tabular} 
\end{table}

\newpage

\begin{table}[H]
\normalsize % The size of the table text can be changed depending on content.
\begin{tabular}{ | c | c | }
\hline
\textbf{Week} & \textbf{Content} \\
\hline

10: 20 Mar.--22 Mar. & \begin{minipage}{.85\textwidth}
\begin{itemize} \itemsep-0.4em
  \vspace{1mm}
  \item \textit{Topics:} Pathwise differentiable target parameters, gradients
    and the canonical gradient of infinite-dimensional models
  \item \textit{Before:} Read A.~3 of vdL\&R (2011); watch Week 10 video
    lectures
  \item \textit{Lab:} Estimation and inference with (efficient) influence
    functions, part 2
  \item \textit{Deliverables:} Final project proposals due 22 Mar.~by 11:59pm,
    Midterm feedback survey due 23 Mar.~by 11:59pm
  \vspace{1mm}
\end{itemize}
\end{minipage} \\
\hline

11: 03 Apr.--05 Apr. & \begin{minipage}{.85\textwidth}
\begin{itemize} \itemsep-0.4em
  \vspace{1mm}
  \item \textit{Topics:} Definition of MLEs and NP-MLEs, efficient influence
    curves, theorems of efficiency
  \item \textit{Before:} Read A.~4 of vdL\&R (2011); watch Week 11 video
    lectures
  \item \textit{Lab:} Super Learning for surival analysis
  \item \textit{Deliverables:} Homework assignment 4 due 05 Apr.~by 11:59pm
  \vspace{1mm}
\end{itemize}
\end{minipage} \\
\hline

12: 10 Apr.--12 Apr. & \begin{minipage}{.85\textwidth}
\begin{itemize} \itemsep-0.4em
  \vspace{1mm}
  \item \textit{Topics:} Efficient one-step estimators, online one-step
    estimators
  \item \textit{Before:} Read Ch.4--6 and A.~6 of vdL\&R (2011); watch Week 12
    video lectures
  \item \textit{Lab:} Computation of Targeted Maximum Likelihood Estimators
  \vspace{1mm}
\end{itemize}
\end{minipage} \\
\hline

13: 17 Apr.--19 Apr. & \begin{minipage}{.85\textwidth}
\begin{itemize} \itemsep-0.4em
  \vspace{1mm}
  \item \textit{Topics:} Targeted Maximum Likelihood Estimation (TMLE)
  \item \textit{Before:} Read Ch.~3--4 of vdL\&R (2017); watch Week 13 video
    lectures
  \item \textit{Lab:} Targeted Learning for surival analysis with the
    {\rm survtmle} R package
  \item \textit{Deliverables:} Homework assignment 5 due 19 Apr.~by 11:59pm,
    Blog question due 20 Apr.~by 11:59pm
  \vspace{1mm}
\end{itemize}
\end{minipage} \\
\hline

14: 24 Apr.--26 Apr. & \begin{minipage}{.85\textwidth}
\begin{itemize} \itemsep-0.4em
  \vspace{1mm}
  \item \textit{Topics:} TMLEs of causal effects of multiple time-point
    interventions based on longitudinal data
  \item \textit{Before:} Watch Week 14 video lectures
  \item \textit{Lab:} Longitudinal Targeted Maximum Likelihood Estimation with
    the {rm stremr} R package
  \vspace{1mm}
\end{itemize}
\end{minipage} \\
\hline

15: 01 May--03 May & \begin{minipage}{.85\textwidth}
\begin{itemize} \itemsep-0.4em
  \vspace{1mm}
  \item \textit{Topics:} RRR week --- Final project presentations
  \item \textit{Deliverables:} Final Project Presentations on 03 May, time TBA;
    Final Project Reports due 04 May by 5:00pm
  \vspace{1mm}
\end{itemize}
\end{minipage} \\
\hline

\end{tabular} 
\end{table}

\end{document}

